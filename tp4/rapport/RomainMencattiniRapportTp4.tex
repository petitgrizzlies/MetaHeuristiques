\documentclass[a4paper, 11pt]{article}

\usepackage[utf8]{inputenc}
\usepackage[T1]{fontenc}
\usepackage[francais]{babel} 
\usepackage{amsmath} % pour les formules de maths
\usepackage{amssymb} % pour des symboles
\usepackage{mathrsfs} % pour avoir acces a des jolies lettres calligrafiées. :)
\usepackage{listings} % pour le code source
\usepackage{color} % pour les couleurs
\usepackage{graphicx} % pour les graphiques (images)
\usepackage{fancyhdr} % pour utiliser le pagestyle fancy
\usepackage[headheight=14pt]{geometry} % pour les marges

\geometry{hmargin=3cm}

\title{Metaheuristics for Optimization\\Series 4: Ant System and Traveling Salesman Problem}
\author{Romain Mencattini}
\date{\today}

\pagestyle{fancy} % pour avoir des entetes et des pieds de page
\renewcommand\headrulewidth{0.6pt}
\fancyhead[L]{Romain Mencattini} % haut de page gauche
\fancyhead[R]{Université de Genève \today} % haut de page droite

\begin{document}
\maketitle
\newpage
\tableofcontents
\newpage

\section{Le voyageur de commerce}
Il s'agit du problème sur lequel nous allons tester l'algorithme du recuit simulé.\\
Détaillons le de manière mathématique.\\
Nous avons un ensemble de villes : $V$. Où $v = \{x,y\} \in V$. est une ville qui avec une coordonnée $x$ et une coordonnée $y$ qui appratiennent 
à $\mathbb{N}$.

On veut visiter toutes les villes, une et une seule fois. Notre résultat sera une permutation de ces villes: $[v_1,v_2,...,v_N]$.
Une liste avec $N$ villes. Où $v_1$ 
sera la première ville visitée, $v_2$ la deuxième etc.

On utilise la distance euclidienne :
\begin{center}
\begin{math}
 euclidienne(v_1,v_2) = \sqrt{(v_{1_x} - v_{2_x})^2 + (v_{1_y} - v_{2_y})^2}
\end{math} 
\end{center}

On veut minimiser la distance totale qui est:
\begin{center}
 \begin{math}
  d_{total} = \sum\limits_{i=2}^{N} euclidienne(v_{i-1},v_i)
 \end{math}
\end{center}

\section{Algorithme}

Pour ce faire nous allons comparer plusieurs méthodes:
\begin{itemize}
 \item Greedy
 \item Ant System séquentiel
 \item Ant System parallèle
\end{itemize}

\subsection{Greedy}
\subsubsection{Algorithme}
L'algorithme est relativement simple. C'est le même que celui programmé dans le tp3.
Il faut:
\begin{enumerate}
 \item Choisir une ville aléatoire, $v_0$.
 \item Ajouter la ville la plus prochaine à notre solution.
 \item Répéter l'étape 2. jusqu'à ce qu'on ait vue toutes les viles.
\end{enumerate}

Suivant le choix du développeur, on peut choisir d'ajouter la première ville, à la fin pour avoir une boucle.
Cet algorithme est totalement déterministe pour un $V$ fixé et une ville initiale $v_0$ fixée.

\subsubsection{Mesures}
Voici les résultats obtenus en moyenne lors de vingt lancés:
\paragraph{}
\begin{tabular}{|l | c |c |}
\hline
 File & Mean & Std \\
\hline
./data/cities.dat	& 31.537453312384304  & 1.8045784058777552\\
\hline
./data/cities2.dat	& 3.387658063949946  & 0.2163689591198703\\
\hline
./data/cities50.dat	& 6.278780156847236  & 2.332163876429746e-15\\
\hline
./data/cities60.dat	& 6.302401870939252  & 0.047031657572111\\
\hline
./data/cities80.dat	& 6.2813443585626185  & 1.2394364910839084e-15\\
\hline
./data/cities100.dat	& 6.282070156684796  & 1.8110771949088597e-15\\
\hline
\end{tabular}
\paragraph{}
On remarque que l'algorithme trouve toujours la solution optimale pour le cercle. Ce qui est évident. En prenant le plus proche voisins,
on trouve notre point à gauche (ou à droite), et ainsi de suite jusqu'à faire le tour du cercle.
Il y a une déviation standard assez importante pour les deux premiers fichiers. Cela vient du faire que l'algorithme est déterministe, donc
il dépend du point de départ. Ce qui n'est pas le cas des algorithmes fourmis.

\subsection{Ant System Séquentiel}
Les deux versions de cet algorithme (séquentielle et parallèle) sont très proches, nous allons donc expliquer en premier lieu 
la version séquentielle plus parallèle.

\subsubsection{Algorithme}
Voici le déroulement de l'algorithme:
\begin{enumerate}
 \item on choisit une ville $v_0$ de départ pour la fourmis.
 \item tant qu'il existe des villes non visitées, on en choisit une $v_i$ et on l'ajoute.
 \item marquer le chemin.
\end{enumerate}

On répète cela pour chacune des $m$ fourmis. Cela correspond à une itération.
Une fois l'itération terminée, on met à jour la matrice des phéromones et on reprend les itérations jusqu'à en avoir exécuté $t_{max}$.

Nous allons maintenant expliquer les points cités.
Pour le choix de la fourmis, nous les avons répartis une par ville, (en utilisant le modulo). Il faut donc que $n$ le nombre de villes, 
soit égal à $m$ le nombre de fourmis. C'est dans ces conditions que les meilleurs résultats sont atteints.

Pour choisir parmis les villes, nous générons un tableau de probabilités (associée une probabilité à chaque ville),et nous choisissons 
la ville suivante en suivant ces probabilité.
Voici comme elles sont générées:
 \begin{center}
  \begin{displaymath}
    p_{ij}^k(t) = 
    \left\lbrace
      \begin{array}{ll}
	\frac{(\tau_{ij}(t))^{\alpha} (\eta_{ij})^{\beta}}{\sum_{l \in J} (\tau_{ij}(t))^{\alpha} (\eta_{ij})^{\beta}} & \text{si } 
	j \in J \\
	0 & \text{sinon } \\
      \end{array}
    \right.
\end{displaymath}
 \end{center}

Avec $k$, la $k^{\text{ième}}$ fourmis, $i$ la ville $i$, $j$ la ville $j$, et $\alpha$,$\beta$ des paramètres, et $t$ l'itération courrante.
On va donc favoriser avec $\tau$ les chemins avec beaucoup de phéromones; et avec $\eta$ les villes les plus proches.
Où $\tau$ est la matrice des phéromones, et $\eta$ la matrice de l'inverse des distances entre villes. Donc $\eta_{ij} = \frac{1}{d_{ij}}$

Chaque fourmis va marquer le chemin de cette manière:
 \begin{center}
  \begin{displaymath}
    \Delta\tau_{ij}^k(t) = 
    \left\lbrace
      \begin{array}{ll}
	\frac{Q}{L^k(t)} & \text{si la fourmis } k \text{utilise l'arc} (i,j) \text{dans son parcour}\\
	0 & \text{sinon } \\
      \end{array}
    \right.
\end{displaymath}
 \end{center}

 Où $Q$ est un paramètre. $L^k(t)$ est la longueur du chemine de la fourmis $k$ à l'itération $t$.
Mettre à jour la matrice des phéromones, soit $\tau$, se fait comme suit:
\begin{center}
 \begin{math}
  \tau_{ij}(t+1) = (1-\rho)\tau_{ij}(t) + \sum\limits_{k=1}^{m} \Delta\tau_{ij}^k(t)
 \end{math}
\end{center}

\subsubsection{Mesures}
Pour la version séquentielle, voici les résultats, avec $t_{max} = 50$ et $\rho = 0.1$, toujours vingts itérations:
\paragraph{}
\begin{tabular}{|l|c|c|}
 \hline
 File & Mean & Std\\
 \hline
cities.dat	& 27.86711241242947  & 0.7484163226353873 \\
 \hline
 cities2.dat	& 3.0318633694064667  & 0.04990194967224049\\
 \hline
cities50.dat	& 6.3283035636850675  & 0\\
 \hline
cities60.dat	& 6.2759356854396  & 0\\
 \hline
cities80.dat	& 6.348320460878663  & 0.007431478876196535\\
 \hline
cities100.dat	& 6.282070156684795  & 9.362222582871203e-16\\
 \hline
\end{tabular}

\paragraph{}
On remarque que les résultats sont meilleurs que le Greedy. Ce qui est une bonne chose. L'AS trouve le maximum global du cercle, et fait 
de meilleurs résultats pour \textit{cities.dat} et \textit{cities2.dat}.

\subsection{Ant System Parallèle}
\subsubsection{Algorithme}
La version parallèle est la même que la version séquentielle, sauf qu'on parallélise les $m$ fourmis lors des itérations. En utilisant
les facilités du langage Golang. Il faut ajouter des verrous lors de l'écriture du $\Delta\tau$, et un mécanisme de synchronisation
à la fin d'une itération.
Une fois cela fait, on obtient les mêmes résultats que la version séquentielle. Cependant en plus rapide, il est donc possible d'augmenter
$t_{max}$ dans la version parallèle.

\subsubsection{Mesures}
Voici les mesures pour $t_{max} = 100$ et $\rho = 0.1$, toujours vingts itérations:
\paragraph{}
\begin{tabular}{|l|c|c|}
 \hline
 File & Mean & Std\\
 \hline
 cities.dat& 27.51536258209029 & 0\\
 \hline
cities2.dat& 2.9352583475370393  & 0.03282959295749493\\
 \hline
cities50.dat& 6.278780156847236  & 9.362222582871203e-16\\
 \hline
cities60.dat& 6.275935685439598  & 9.362222582871203e-16\\
 \hline
cities80.dat& 6.281344358562613  & 0\\
 \hline
cities100.dat& 6.282070156684793  & 9.362222582871203e-16\\
 \hline
\end{tabular}

On remarque que les résultats sont meilleurs que la version à 50 itérations. En augmentant la valeur de $t_{max}$, on augmente la qualité
des résultats. En utilisant la version parallèle, nous avons dessiné des exemples de résultats pour avoir une idée:

\includegraphics[scale=0.4]{citieAnt}\\
\includegraphics[scale=0.4]{citieAnt2}\\
\includegraphics[scale=0.4]{citieAnt50}\\
\includegraphics[scale=0.4]{citieAnt60}\\
\includegraphics[scale=0.4]{citieAnt80}\\
\includegraphics[scale=0.4]{citieAnt100}\\

On confirme ce que l'on a compris grâce au tableau de résultats, ci-dessus:
\begin{itemize}
 \item pour le cercle la solution optimale est trouvée
 \item l'algorithme se comporte assez bien même lors de paysage très rugeux (cities2.dat)
 \item il trouve de bonnes valeurs et ce souvent (cities.dat)
\end{itemize}

Pour appuyer ces résultats, nous allons présenter les histogrammes des valeures obtenues pour vingt lancés:\\

\paragraph{}
\textbf{Cities.dat}\\
\includegraphics[scale=0.4]{cities_dat}\\
\paragraph{}\textbf{Cities2.dat}\\
\includegraphics[scale=0.4]{cities2_dat}\\
\paragraph{}\textbf{Cities50.dat}\\
\includegraphics[scale=0.4]{cities50_dat}\\
\paragraph{}\textbf{Cities80.dat}\\
\includegraphics[scale=0.4]{cities80_dat}\\
\paragraph{}\textbf{Cities100.dat}\\
\includegraphics[scale=0.4]{cities100_dat}\\

\paragraph{}Ce qui confirme les résultats.
\end{document}